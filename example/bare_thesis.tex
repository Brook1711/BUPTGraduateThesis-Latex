%%
%% This is file `example/bare_thesis.tex',
%% generated with the docstrip utility.
%%
%% The original source files were:
%%
%% install/buptgraduatethesis.dtx  (with options: `bare-thesis')
%% 
%% This file is a part of the example of BUPTGraduateThesis.
%% 

\documentclass[%
  degree=master,%
  classlevel=open,%
  mathfont=mathptmx,%
  dedication=false,%
  committee=true,%
  chapbib=true,%
  finish=print,%
  driver=xetex]{buptgraduatethesis}

%% 自定义导言区
%% 在这里添加你需要的宏包、自定义命令、环境等
%% \usepackage{...}
%% \DeclareMathOperator{\CT}{H}
%% \DeclareMathOperator{\Cov}{Cov}
\def\BUPTThesis{\textsc{BUPT}\-\textsc{Thesis}}

%% 在这里添加图片文件搜索目录
\graphicspath{{../}}
%% 自定义导言区结束

%% 加载缩略语定义
%%
%% This is file `example/metadata.tex',
%% generated with the docstrip utility.
%%
%% The original source files were:
%%
%% install/buptgraduatethesis.dtx  (with options: `metadata')
%% 
%% This file is a part of the example of BUPTGraduateThesis.
%% 


%% 学号
\studentid{2011000000}

%% 论文题目
\ctitle{北京邮电大学研究生学位论文\LaTeXe{}模板使用示例文档}
\etitle{Example of BUPT Graduate Thesis \LaTeXe{} Template With longlonglong Title}

%% 申请学位
\cdegree{工学硕士}

%% 院系名称
\cdepartment{信息与通信工程学院}
\edepartment{School of Information and Communication Engineering}

%% 专业名称
\cmajor{移动通信}
\emajor{Mobile Communication}

%% 你的姓名
\cauthor{学生姓名}
\eauthor{Pinyin Name}

%% 博士后研究工作报告-分类号
\classnumber{O441.3}

%% 博士后研究工作报告-UDC
\udc{621.396.9}

%% 博士后研究工作报告-学校编号
\schoolserial{147227}

%% 博士后研究工作起始时间
\startdate{2014年10月29日}

%% 博士后研究工作期满时间
\finishdate{2016年4月2日}

%% 培养方式:填写内容应为“全日制”或“非全日制”。
\ctrainingmode{全日制 或 非全日制}
\etrainingmode{Part-Time or Full-Time}

%% 你导师的姓名
\csupervisor{导师姓名}
\esupervisor{Supervisor Name}

%% 中文摘要
\cabstract{%
  中、英文摘要位于声明的次页,摘要应简明表达学位论文的内容要点,体现研究工作的核心思想。
  重点说明本项科研的目的和意义、研究方法、研究成果、结论,注意突出具有创新性的成果和新见解的部分。

  关键词是为文献标引工作而从论文中选取出来的、用以表示全文主题内容信息的术语。
  关键词排列在摘要内容的左下方,具体关键词之间以均匀间隔分开排列,无需其它符号。
}

%% 中文关键词,关键词之间用 \kwsep 分割
\ckeywords{\TeX \kwsep \LaTeX \kwsep xeCJK \kwsep 模板 \kwsep 排版 \kwsep 论文}

%% 英文摘要
\eabstract{%
  The Chinese and English abstract should appear after the declaration page.
  The abstract should present the core of the research work, especially the purpose and importance of the research, the method adopted, the results, and the conclusion.

  Key words are terms selected for documentation indexing, which should present the main contributions of the thesis.
  Key words are aligned at the bottom left side of the abstract content.
  Key words should be seperated by spaces but not any other symbols.
}

%% 英文关键词,也用 \kwsep 分割
\ekeywords{%
  \TeX \kwsep \LaTeX \kwsep xeCJK \kwsep template \kwsep typesetting \kwsep thesis}


\loadglsentries{acronyms}

%% 攻读学位期间发表论文
%% 用 \newcite{<suffix>}{<caption>} 声明不同的论文类型(例如: 期刊论文、会议论文等)。每一个类型的对应的 .bib 文件用 \bibliography<suffix> 命令加载,用 \nocite<suffix> 命令引用。具体请参考 pubs.tex 中的示例
\newcite{jrnl}{期刊论文}
\newcite{conf}{会议论文}
\newcite{patent}{专利}

\begin{document}
%% 声明前置部分
\makefrontmatter

%% 生成主要符号对照表
%%
%% This is file `example/notations.tex',
%% generated with the docstrip utility.
%%
%% The original source files were:
%%
%% install/buptgraduatethesis.dtx  (with options: `notations')
%% 
%% This file is a part of the example of BUPTGraduateThesis.
%% 

\begin{listofnotations}
\item [$(\cdot)^*$] 复共轭
\item [$(\cdot)^{\mathrm T}$] 矩阵转置
\item [$(\cdot)^{\mathrm H}$] 矩阵共轭转置
\item [$\mathbf{X}$] 矩阵或向量
\item [$\mathcal{A}$] 集合
\item [$\mathcal{A}\times\mathcal{B}$]
  集合 $\mathcal{A}$ 与集合 $\mathcal{B}$ 的 Cartesian 积,即 $\mathcal{A}\times\mathcal{B}=\{(a,b):a\in\mathcal{A},b\in\mathcal{B}\}$
\end{listofnotations}


%% 主体部分
\mainmatter
%% 用\include{}命令引用各章.tex文件
\include{ch_intro}
\include{ch_concln}

\ifx\usechapbib\undefined
\bibliographystyle{buptgraduatethesis}
\bibliography{bare_thesis}
\fi

%% 附录部分

%% 如果有两个或两个以上的附录, 使用appendix环境
\begin{appendix}
  \input{app_lhospital}
  % 自动抽取生成缩略语表作为附录A
  \tableofacronyms
  % 用\input{}添加其他的附录
  % \input{...}
\end{appendix}

%% 如果只有一个附录, 使用appendix*环境
%% \begin{appendix*}
%%   % 自动抽取生成缩略语表作为附录A
%%   % \tableofacronyms
%% \end{appendix*}

\backmatter
%% 致谢
\ifx\ispeerreview\undefined
%%
%% This is file `example/ackgmt.tex',
%% generated with the docstrip utility.
%%
%% The original source files were:
%%
%% install/buptgraduatethesis.dtx  (with options: `ackgmt')
%% 
%% This file is a part of the example of BUPTGraduateThesis.
%% 

\begin{acknowledgement}
  %% 感谢所有你应该感谢的人
  感谢Donald Ervin Knuth.
\end{acknowledgement}

\fi

%% 在读期间论文发表情况
%%
%% This is file `example/pubs.tex',
%% generated with the docstrip utility.
%%
%% The original source files were:
%%
%% install/buptgraduatethesis.dtx  (with options: `pubs')
%% 
%% This file is a part of the example of BUPTGraduateThesis.
%% 

%% 发表论文列表

%% 攻读学位期间发表论文列表用 tableofpublications 环境产生。需要
%% 在 bare_thesis.tex 的导言区用 \newcite{<name>}{<caption>} 声明不同类
%% 型的论文,具体见导言区说明。
%% 根据各类论文发表数量设置\setbiblabelwidth{<num>},用于控制发表论文序号的对齐位置。
%% 例如:发表conf类论文数量为个位数,则<num>=1;发表jrnl类论文数量为两位数,则<num>=10;
\begin{tableofpublications}
  \thispagestyle{bupt@pubheadings}%
  \setcounter{NAT@ctr}{0}
  \setbiblabelwidth{1}
  \bibliographyjrnl{pubs}
  \nocitejrnl{paper1}

  \setbiblabelwidth{1}
  \bibliographyconf{pubs}
  \nociteconf{paper2}

  \setbiblabelwidth{1}
  \bibliographypatent{pubs}
  \nocitepatent{patent1}
\end{tableofpublications}


\newpage
\end{document}
